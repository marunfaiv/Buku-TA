\begin{center}
  \large\textbf{ABSTRACT}
\end{center}

\addcontentsline{toc}{chapter}{ABSTRACT}

\vspace{2ex}

\begingroup
% Menghilangkan padding
\setlength{\tabcolsep}{0pt}

\noindent
\begin{tabularx}{\textwidth}{l >{\centering}m{3em} X}
  \emph{Name}     & : & \name{}         \\

  \emph{Title}    & : & \engtatitle{}   \\

  \emph{Advisors} & : & 1. \advisor{}   \\
                  &   & 2. \coadvisor{} \\
\end{tabularx}
\endgroup

% Ubah paragraf berikut dengan abstrak dari tugas akhir dalam Bahasa Inggris
The emergence of Non-Fungible Tokens (NFTs) as a key component of the digital economy has sparked significant interest in their potential to revolutionize various industries, from art and entertainment to digital identity management. A fundamental aspect of maximizing the utility of NFTs involves ensuring their seamless operation across different blockchain platforms, which is addressed by the concept of interoperability. This thesis presents a comprehensive study and implementation of blockchain-based NFT interoperability using smart contracts within the framework of Web3.0. The research primarily focuses on developing a smart contract architecture that not only supports the standard features of NFTs but also facilitates their cross-chain interactions. By leveraging the Ethereum blockchain and utilizing the ERC-721 standard, this work establishes a robust protocol for NFT creation, transaction, and management that ensures secure, transparent, and efficient interoperability. Key aspects include the design of a decentralized application (DApp) that interacts with smart contracts to mint, manage, and transfer NFTs across blockchain boundaries. The implementation demonstrates the practicality and efficiency of the proposed system in a controlled testnet environment. Future work might explore scaling solutions, enhanced security features, and the integration of additional blockchain platforms to extend the reach and applicability of interoperable NFTs in the expanding universe of Web3.0.

% Ubah kata-kata berikut dengan kata kunci dari tugas akhir dalam Bahasa Inggris
Keywords: Non-Fungible Tokens (NFTs), Blockchain Interoperability, Smart Contracts, Web3.0, Ethereum
