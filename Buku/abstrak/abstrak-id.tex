\begin{center}
  \large\textbf{ABSTRAK}
\end{center}

\addcontentsline{toc}{chapter}{ABSTRAK}

\vspace{2ex}

\begingroup
% Menghilangkan padding
\setlength{\tabcolsep}{0pt}

\noindent
\begin{tabularx}{\textwidth}{l >{\centering}m{2em} X}
  Nama Mahasiswa    & : & \name{}         \\

  Judul Tugas Akhir & : & \tatitle{}      \\

  Pembimbing        & : & 1. \advisor{}   \\
                    &   & 2. \coadvisor{} \\
\end{tabularx}
\endgroup

% Ubah paragraf berikut dengan abstrak dari tugas akhir
Emergensi \emph{Non-Fungible Token} (NFT) sebagai komponen kunci dalam ekonomi digital telah memicu minat signifikan terhadap potensinya untuk merevolusi berbagai industri, mulai darian hiburan hingga manajemen identitas digital. Aspek fundamental dalam memaksimalkan kegunaan NFT melibatkan kepastian operasionalnya lintas platform \emph{blockchain}, yang diatasi dengan konsep interoperabilitas. Tesis ini menyajikan studi komprehensif dan implementasi interoperabilitas NFT berbasis \emph{blockchain} menggunakan \emph{smart contract} dalam kerangka Web3.0. Riset ini terutama fokus pada pengembangan \emph{smart contract} yang tidak hanya mendukung fitur standar dari NFT tetapi juga memfasilitasi interaksi lintas rantai mereka. Dengan memanfaatkan \emph{blockchain} Ethereum dan menggunakan standar ERC-721, pekerjaan ini mendirikan protokol yang kuat untuk pembuatan, transaksi, dan manajemen NFT yang menjamin interoperabilitas yang aman, transparan, danAspek kunci termasuk desain aplikasi terdesentralisasi (DApp) yang berinteraksi dengan \emph{smart contract} untuk mencetak, mengelola, dan mentransfer NFT melintasi batas \emph{blockchain}. Implementasi menunjukkan praktikalitas dan efisiensi sistem yangalam lingkungan testnet terkontrol. Pekerjaan masa depan mungkin mengeksplorasi solusi penskalaan, fitur keamanan yang ditingkatkan, dan integrasi platform \emph{blockchain} tambahan untuk memperluas jangkauan dan aplikabilitas NFT yang interoperabel di alam semesta yang berkembang dari Web3.0.

% Ubah kata-kata berikut dengan kata kunci dari tugas akhir
Kata Kunci: \emph{Non-Fungible Token} (NFTs), \emph{Blockchain}, \emph{Interoperability}, \emph{Smart Contracts}, Web3.0, Ethereum
