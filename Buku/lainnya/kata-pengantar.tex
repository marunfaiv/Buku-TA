\begin{center}
  \Large
  \textbf{KATA PENGANTAR}
\end{center}

\addcontentsline{toc}{chapter}{KATA PENGANTAR}

\vspace{2ex}

% Ubah paragraf-paragraf berikut dengan isi dari kata pengantar

Puji dan syukur kehadirat  Allah SWT. yang telah melimpahkan rahmat dan hidayah-Nya sehingga penyusun dapat menyelesaikan laporan skripsi berjudul Interoperabilitas NFT Berbasis Blockchain Menggunakan Smart Contract Pada Metaverse dan Web3.0.

Penelitian ini disusun dalam rangka sebagai salah satu syarat untuk memperoleh gelar Sarjana Teknik Komputer Program Studi Teknik Komputer Institut Teknologi Sepuluh Nopember. Laporan ini berisi hasil penelitian yang telah dilakukan oleh penyusun selama kurang lebih enam bulan.
Oleh karena itu, penulis mengucapkan terima kasih kepada:

\begin{enumerate}[nolistsep]

  \item Keluarga, Ibu, Bapak dan Saudara tercinta yang telah memberikan dukungan dalam penyelesaian tugas akhir ini.

  \item Bapak Mochamad Hariadi, S.T., M.Sc., Ph.D, dan bapak Reza Fuad Rachmadi, S.T., M.T., Ph.D selaku dosen pembimbing satu dan kedua yang telah membimbing saya dalam menyelesaikan tugas akhir ini.

  \item \lipsum[5][1-3]

\end{enumerate}

Akhir kata, penyusun menyadari bahwa laporan skripsi ini masih jauh dari kata sempurna. Dengan demikian, penyusun mengharapkan saran dan kritik yang membangun dari pembaca. Semoga laporan ini dapat bermanfaat bagi para pendidik dan mahasiswa dalam meningkatkan literasi dan pengetahuan mengenai Smart Contract, Blockchain, serta NFT.

\begin{flushright}
  \begin{tabular}[b]{c}
    \place{}, \MONTH{} \the\year{} \\
    \\
    \\
    \\
    \\
    \name{}
  \end{tabular}
\end{flushright}
