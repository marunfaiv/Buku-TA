\chapter{PENUTUP}
\label{chap:penutup}

% Ubah bagian-bagian berikut dengan isi dari penutup

\section{Kesimpulan}
\label{sec:kesimpulan}

Dalam penelitian ini, penulis telah mengeksplorasi dan mengimplementasikan konsep interoperabilitas NFT berbasis blockchain menggunakan smart contract dalam konteks Web3.0. Studi ini berhasil menunjukkan potensi dan efektivitas teknologi blockchain, khususnya Ethereum, dalam mendukung penciptaan dan pengelolaan NFT yang dapat beroperasi lintas berbagai token. Lalu Kesimpulan dari penelitian ini adalah
\begin{itemize}
\item Implementasi Smart contract yang dirancang telah berhasil memfasilitasi penciptaan, transaksi, dan verifikasi NFT secara aman dan efisien. Dengan menggunakan standar ERC-721, penulis telah mengimplementasikan NFT yang unik dan dapat dilacak asal-usulnya, yang sangat penting dalam ekosistem Web3.0.
\item NFT yang dikembangkan menunjukkan interoperabilitas lintas berbagai platform \emph{blockchain} Hal ini dicapai melalui penggunaan protokol \emph{cross-chain interoperability protocol} yang konsisten dengan percobaan transfer \emph{ownership} secara lintas rantai yang dilakukan dengan iterasi 100 kali.  
\item NFT yang dikembangkan berhasil diintegrasikan dengan \emph{network sepolia ethereum} untuk melakukan transaksi pada \emph{network} tersebut. 
\end{itemize}

\section{Saran}
Untuk meningkatkan interoperabilitas dan efisiensi penggunaan NFT d disarankan agar penelitian lebih lanjut difokuskan pada pengembangan protokol yang lebih universal untuk integrasi lintas rantai. Selain itu, penelitian lebih lanjut diperlukan untuk menyempurnakan mekanisme keamanan yang dapat melindungi privasi pengguna sambil mempertahankan transparansi dan keandalan sistem blockchain.