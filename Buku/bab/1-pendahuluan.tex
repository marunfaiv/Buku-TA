\chapter{PENDAHULUAN}
\label{chap:pendahuluan}

% Ubah bagian-bagian berikut dengan isi dari pendahuluan

Penelitian ini dilatarbelakangi oleh \lipsum[1][1-5]

\section{Latar Belakang}
\label{sec:latarbelakang}

Dalam beberapa tahun terakhir, NFT telah menjadi sorotan utama bagi kalangan industri dan akademisi. Data menunjukkan bahwa dalam sehari, volume transaksi pasar NFT mencapai sekitar 4,592 miliar USD, sedangkan total transaksi pasar kripto dalam sehari mencapai sekitar 341,017 miliar USD \parencite{Wang2021}. Non-Fungible Tokens (NFTs) adalah aset digital yang mewakili objek seperti seni, barang koleksi, dan item dalam game. Aset ini diperjualbelikan di internet, kebanyakan dengan mata uang kripto, dan umumnya tertanam dalam \emph{smart contract} di sebuah \emph{blockchain}. NFT juga bersifat khas sehingga tidak dapat dipertukarkan dengan objek serupa, hal ini menjadikannya tepat untuk mengenal pasti sesuatu atau seseorang dengan cara unik. Walaupun NFT menjanjikan dampak signifikan pada pasar yang terdesentralisasi sekarang dan kesempatan bisnis yang akan datang, teknologi NFT sejatinya masih di tahap permulaan. Ada beberapa hambatan yang perlu diatasi dengan seksama, dan sejumlah peluang besar yang perlu diperhatikan \parencite{Khan2021}.

NFT yang didukung oleh teknologi \emph{blockchain} dan \emph{smart contract} menjanjikan peluang besar bagi industri kreatif, meskipun kehadiran mereka telah mengguncang pasar. Namun, NFT juga memiliki keterbatasan yang berasal dari teknologi \emph{blockchain} dasar. Blockchain sendiri memastikan kesepakatan yang terpercaya dalam sistemnya yang terdistribusi dengan mengandalkan komputer-komputer (sering disebut sebagai "miners") untuk menyelesaikan masalah matematika yang kompleks. Salah satu hambatan besar dalam evolusi teknologi \emph{blockchain} adalah interoperabilitas. Walaupun \emph{blockchain} memberikan solusi yang kokoh dan dapat diandalkan, beragam varian dan tipe \emph{blockchain} yang ada kini kerap kali mengalami kesulitan dalam berinteraksi dan berkomunikasi dengan baik. Hal ini mengimplikasikan bahwa \emph{smart contract} atau NFT yang dikembangkan pada suatu \emph{blockchain} mungkin tidak kompatibel atau dikenali oleh blockchain tipe lain. Kendala ini membatasi kemampuan \emph{blockchain} untuk berkembang dan berintegrasi dengan sistem industri yang lebih besar, seperti sektor keuangan, medis, atau bisnis internasional \parencite{Malik2023}.

Blockchain dan Smart Contract menjadi fondasi penting dalam mewujudkan visi Metaverse dan WEB3.0. Blockchain menyediakan transparansi, keamanan, dan kepercayaan dengan menyimpan data dalam jaringan yang terdesentralisasi, sementara Smart Contract memungkinkan otomatisasi transaksi dan perjanjian di dunia digital tanpa perlu perantara. Kombinasi dari teknologi-teknologi ini dapat membentuk fondasi untuk era baru di dunia digital, di mana interaksi lebih aman, transparan, dan tanpa hambatan \parencite{Gadekallu2022}.

\lipsum[2]

\section{Permasalahan}
\label{sec:permasalahan}

% Ubah paragraf berikut sesuai dengan rumusan masalah dari tugas akhir
Berdasarkan latar belakang di atas, maka dapat dirumuskan masalah pada Tugas Akhir ini adalah kekurangan interoperabilitas NFT di antara berbagai platform.

\section{Batasan Masalah atau Ruang Lingkup}
Penelitian ini akan menggunakan platform blockchain Ethereum, mengingat SOLIDITY merupakan bahasa pemrograman khusus untuk Ethereum. Lalu penelitian akan difokuskan pada NFT dalam bentuk properti virtual. Penelitian tidak akan mencakup aspek-aspek hukum atau regulasi terkait transaksi NFT. Evaluasi kinerja akan berfokus pada interoperabilitas dan keandalan \emph{smart contract}, dan tidak mencakup aspek-aspek lain seperti efisiensi atau skalabilitas. Pengujian dan validasi akan dilakukan dalam lingkungan \emph{testnet}, bukan pada jaringan \emph{blockchain} utama atau \emph{mainnet}.

\section{Tujuan}
\label{sec:Tujuan}

% Ubah paragraf berikut sesuai dengan tujuan penelitian dari tugas akhir
Tujuan utama dari penelitian ini adalah 

\begin{enumerate}[nolistsep]

  \item Mengembangkan framework blockchain yang dapat berinteraksi dan berkomunikasi dengan blockchain lain tanpa hambatan

  \item Mengembangkan smart contract yang dirancang khusus untuk memastikan bahwa NFT dari platform lain dapat dikenali, diperdagangkan, dan diintegrasikan dengan mudah

\end{enumerate}

\section{Batasan Masalah}
\label{sec:batasanmasalah}

Batasan-batasan dari penilitian tugas akhir ini terdapat beberapa, yakni adalah:

\begin{enumerate}[nolistsep]

  \item Pembuatan sistem dikerjakan dalam lingkup \emph{testnet}.

  \item \emph{smart contract} dibuat dengan menggunakan bahasa pemrograman Solidity.

  \item Menggunakan Hardhat sebagai \emph{ethereum development environment} dan Metamask sebagai virtual wallet. 

\end{enumerate}

\section{Sistematika Penulisan}
\label{sec:sistematikapenulisan}

Laporan penelitian tugas akhir ini terbagi menjadi \lipsum[1][1-3] yaitu:

\begin{enumerate}[nolistsep]

  \item \textbf{BAB I Pendahuluan}

        Bab ini berisi \lipsum[2][1-5]

        \vspace{2ex}

  \item \textbf{BAB II Tinjauan Pustaka}

        Bab ini berisi \lipsum[3][1-5]

        \vspace{2ex}

  \item \textbf{BAB III Desain dan Implementasi Sistem}

        Bab ini berisi \lipsum[4][1-5]

        \vspace{2ex}

  \item \textbf{BAB IV Pengujian dan Analisa}

        Bab ini berisi \lipsum[5][1-5]

        \vspace{2ex}

  \item \textbf{BAB V Penutup}

        Bab ini berisi \lipsum[6][1-5]

\end{enumerate}
