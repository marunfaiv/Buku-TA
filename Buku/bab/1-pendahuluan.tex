\chapter{PENDAHULUAN}
\label{chap:pendahuluan}

% Ubah bagian-bagian berikut dengan isi dari pendahulua

\section{Latar Belakang}
\label{sec:latarbelakang}

Dalam beberapa tahun terakhir, NFT telah menjadi sorotan utama bagi kalangan industri dan akademisi. Data menunjukkan bahwa dalam sehari, volume transaksi pasar NFT mencapai sekitar 4,592 miliar USD, sedangkan total transaksi pasar kripto dalam sehari mencapai sekitar 341,017 miliar USD \cite{Wang2021}. Non-Fungible Tokens (NFTs) adalah aset digital yang mewakili objek seperti seni, barang koleksi, dan item dalam game. Aset ini diperjualbelikan di internet, kebanyakan dengan mata uang kripto, dan umumnya tertanam dalam \emph{smart contract} di sebuah \emph{blockchain}. NFT juga bersifat khas sehingga tidak dapat dipertukarkan dengan objek serupa, hal ini menjadikannya tepat untuk mengenal pasti sesuatu atau seseorang dengan cara unik. Walaupun NFT menjanjikan dampak signifikan pada pasar yang terdesentralisasi sekarang dan kesempatan bisnis yang akan datang, teknologi NFT sejatinya masih di tahap permulaan. Ada beberapa hambatan yang perlu diatasi dengan seksama, dan sejumlah peluang besar yang perlu diperhatikan \cite{Khan2021}.

NFT yang didukung oleh teknologi \emph{blockchain} dan \emph{smart contract} menjanjikan peluang besar bagi industri kreatif, meskipun kehadiran mereka telah mengguncang pasar. Namun, NFT juga memiliki keterbatasan yang berasal dari teknologi \emph{blockchain} dasar. Blockchain sendiri memastikan kesepakatan yang terpercaya dalam sistemnya yang terdistribusi dengan mengandalkan komputer-komputer (sering disebut sebagai "miners") untuk menyelesaikan masalah matematika yang kompleks. Salah satu hambatan besar dalam evolusi teknologi \emph{blockchain} adalah interoperabilitas. Walaupun \emph{blockchain} memberikan solusi yang kokoh dan dapat diandalkan, beragam varian dan tipe \emph{blockchain} yang ada kini kerap kali mengalami kesulitan dalam berinteraksi dan berkomunikasi dengan baik. Hal ini mengimplikasikan bahwa \emph{smart contract} atau NFT yang dikembangkan pada suatu \emph{blockchain} mungkin tidak kompatibel atau dikenali oleh blockchain tipe lain. Kendala ini membatasi kemampuan \emph{blockchain} untuk berkembang dan berintegrasi dengan sistem industri yang lebih besar, seperti sektor keuangan, medis, atau bisnis internasional \cite{Malik2023}.

Interoperabilitas yang rendah ini menciptakan silo dalam industri blockchain, yang pada gilirannya mengurangi potensi inovasi dan pertukaran nilai secara efisien. Dalam konteks Web 3.0 — era internet yang lebih terdesentralisasi dan pengguna-berkuasa — interoperabilitas menjadi penting untuk memastikan bahwa teknologi blockchain dapat mendukung sebuah ekosistem digital yang benar-benar terbuka dan terhubung. Oleh karena itu, tugas akhir ini bertujuan untuk mengembangkan solusi yang memungkinkan interoperabilitas NFT lintas blockchain, sehingga memfasilitasi transfer dan pengakuan aset digital di berbagai platform blockchain. Solusi ini diharapkan tidak hanya akan mengatasi hambatan teknis saat ini tetapi juga akan membuka pintu untuk adopsi yang lebih luas dari teknologi NFT dalam berbagai aplikasi praktis di Web 3.0, dari seni digital hingga real estate dan beyond.

Blockchain dan Smart Contract menjadi fondasi penting dalam mewujudkan visi Web3.0. Blockchain menyediakan transparansi, keamanan, dan kepercayaan dengan menyimpan data dalam jaringan yang terdesentralisasi, sementara Smart Contract memungkinkan otomatisasi transaksi dan perjanjian di dunia digital tanpa perlu perantara. Kombinasi dari teknologi-teknologi ini dapat membentuk fondasi untuk era baru di dunia digital, di mana interaksi lebih aman, transparan, dan tanpa hambatan \cite{Gadekallu2022}. 

\section{Permasalahan}
\label{sec:permasalahan}
Berdasarkan latar belakang di atas, maka dapat dirumuskan masalah pada tugas akhir ini adalah bagaimana mengembangkan solusi interoperabilitas untuk \emph{Non-Fungible Tokens} (NFT) yang memungkinkan transfer lintas berbagai platform \emph{blockchain} dengan efisien dan aman.

\section{Batasan Masalah atau Ruang Lingkup}
\label{sec:batasanmasalah}
\begin{enumerate}[nolistsep]

  \item Penelitian ini akan menggunakan platform \emph{blockchain} Ethereum, mengingat Solidity merupakan bahasa pemrograman khusus untuk Ethereum.

  \item Penelitian akan difokuskan pada NFT dalam bentuk gambar.
  
  \item Pengujian dan validasi akan dilakukan dalam lingkungan \emph{testnet}, bukan pada jaringan \emph{blockchain} utama atau \emph{mainnet}.

\end{enumerate}

\section{Tujuan}
\label{sec:Tujuan}

% Ubah paragraf berikut sesuai dengan tujuan penelitian dari tugas akhir
Tujuan utama dari penelitian ini adalah 

\begin{enumerate}[nolistsep]

  \item Mengembangkan sistem \emph{smart contract} yang dapat berinteraksi dan berkomunikasi dengan \emph{blockchain} pada \emph{network} lain tanpa hambatan.

  \item Mengimplementasikan sistem \emph{smart contract} yang berinteroperabilitas dengan web3.0.

\end{enumerate}

\section{Manfaat}
\label{sec:manfaat}

\begin{enumerate}[nolistsep]

  \item Riset ini memfasilitasi pertukaran dan pemanfaatan token non-fungible (NFT) secara mulus lintas berbagai \emph{network} \emph{blockchain}. Dengan mengatasi masalah interoperabilitas, ini memungkinkan pemilik NFT menggunakan aset mereka di berbagai lingkungan aplikasi terdesentralisasi tanpa terikat pada satu jaringan \emph{blockchain} saja.

  \item Penggunaan \emph{smart contract} dalam mengelola transaksi NFT mengenalkan pendekatan standarisasi yang meningkatkan keamanan, transparansi, dan efisiensi. 

  \item Dengan mengintegrasikan NFT dengan Web3.0, tugas akhir ini mendorong batasan cara aplikasi terdesentralisasi dapat beroperasi, mengarah pada platform yang lebih otonom, tanpa kepercayaan, dan diatur oleh pengguna.

\end{enumerate}

\section*{Sistematika Penulisan}
Laporan penelitian tugas akhir ini terbagi menjadi 5 bab yaitu:

\begin{enumerate}
  \item \textbf{BAB I Pendahuluan}
  \\ Bab ini berisi latar belakang, permasalahan, batasan masalah atau ruang lingkup, tujuan, manfaat, dan juga sistematika penulisan dari penelitian ini.
  \item \textbf{BAB II Tinjauan Pustaka}
  \\ Bab ini berisi penelitian-penelitian terdahulu yang relevan dengan penelitian ini. Dalam bab ini juga dijelaskan teori-teori yang mendukung penelitian ini.
  \item \textbf{BAB III Metodologi}
  \\ Bab ini berisi keperluan yang digunakan dalam penelitian ini, termasuk metode, alat yang digunakan serta urutan dalam pelaksanaan penelitian.
  \item \textbf{BAB IV Hasil dan Pembahasan}
  \\ Bab ini berisi hasil dari pengujian-pengujian yang sudah dilakukan dan hasil yang didapatkan dari setiap pengujian yang sudah dilakukan
  \item \textbf{BAB V Penutup}
  \\ Bab ini berisi kesimpulan serta hasil akhir dari penelitian ini, dan juga menjawab permasalahan yang sudah dijelaskan di bab pendahuluan.
\end{enumerate}
