% Atur variabel berikut sesuai namanya

% nama
\newcommand{\name}{Arya Abdul Azis}
\newcommand{\authorname}{Azis, Arya Abdul}
\newcommand{\nickname}{Arya}
\newcommand{\advisor}{Mochamad Hariadi, S.T., M.Sc., Ph.D}
\newcommand{\coadvisor}{Reza Fuad Rachmadi, S.T., M.T., Ph.D}
\newcommand{\examinerone}{Dr. Galileo Galilei, S.T., M.Sc}
\newcommand{\examinertwo}{Friedrich Nietzsche, S.T., M.Sc}
\newcommand{\examinerthree}{Alan Turing, ST., MT}
\newcommand{\headofdepartment}{Dr. Supeno Mardi Susiki Nugroho,S.T.,M.T.}

% identitas
\newcommand{\nrp}{5024201069}
\newcommand{\advisornip}{199691209199703 1 002}
\newcommand{\coadvisornip}{19850403201212 1 001}
\newcommand{\examineronenip}{18560710 194301 1 001}
\newcommand{\examinertwonip}{18560710 194301 1 001}
\newcommand{\examinerthreenip}{18560710 194301 1 001}
\newcommand{\headofdepartmentnip}{19700313199512 1 001}

% judul
\newcommand{\tatitle}{INTEROPERABILITAS NFT BERBASIS BLOCKCHAIN MENGGUNAKAN SMART CONTRACT PADA WEB3.0}
\newcommand{\engtatitle}{\emph{BLOCKCHAIN-BASED NFT INTEROPERABILITY USING SMART
CONTRACTS IN WEB3.0}}

% tempat
\newcommand{\place}{Surabaya}

% jurusan
\newcommand{\studyprogram}{Teknik Komputer}
\newcommand{\engstudyprogram}{Computer Engineering}

% fakultas
\newcommand{\faculty}{Teknologi Elektro dan Informatika Cerdas}
\newcommand{\engfaculty}{Intelligent Electrical and Informatics Technology}

% singkatan fakultas
\newcommand{\facultyshort}{FTEIC}
\newcommand{\engfacultyshort}{FTEIC}

% departemen
\newcommand{\department}{Teknik Komputer}
\newcommand{\engdepartment}{Computer Engineering}

% kode mata kuliah
\newcommand{\coursecode}{EC234801}
