% Ubah judul dan label berikut sesuai dengan yang diinginkan.
\section{Conclusion}
\label{sec:kesimpulan}

% Ubah paragraf-paragraf pada bagian ini sesuai dengan yang diinginkan.
In this research, we have explored and implemented the concept of blockchain-based NFT interoperability using smart contracts within the context of the Metaverse and Web3.0. This study has successfully demonstrated the potential and effectiveness of blockchain technology, particularly Ethereum, in supporting the creation and management of NFTs that can operate across various tokens.
\subsubsection{Conclusions from this research are}
\begin{itemize}
    \item Smart Contract Implementation: The designed smart contracts have successfully facilitated the creation, transactions, and verification of NFTs securely and efficiently. By using the ERC-721 standard, we have implemented unique NFTs that can be traced to their origin, which is essential in the Metaverse and Web3.0 ecosystems.
    \item The developed NFTs show high interoperability across various Metaverse platforms. This was achieved through the use of consistent protocols and APIs, allowing the NFTs to be used in various applications and games within the Web3.0 ecosystem without substantial modifications.

\subsubsection{}{Recommendations for further development}
To enhance the interoperability and efficiency of NFT usage in the Metaverse, it is recommended that future research focus on developing more universal protocols for cross-chain integration. Additionally, further research is needed to refine security mechanisms that can protect user privacy while maintaining the transparency and reliability of the blockchain system.
\end{itemize} 

