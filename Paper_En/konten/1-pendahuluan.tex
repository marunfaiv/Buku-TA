% Ubah judul dan label berikut sesuai dengan yang diinginkan.
\section{Introduction}
\label{sec:pendahuluan}


In recent years, NFTs have come into the spotlight for both industry and academia. Data indicates that the daily transaction volume of the NFT market reaches approximately \$4.592 billion USD, while the total daily transaction volume of the crypto market is around \$341.017 billion USD. Non-Fungible Tokens (NFTs) are digital assets that represent objects like art, collectibles, and in-game items. These assets are traded over the internet, mostly with cryptocurrencies, and are typically embedded in smart contracts on a blockchain. NFTs are unique, making them non-interchangeable with similar objects, which makes them ideal for uniquely identifying something or someone. Although NFTs promise a significant impact on the current decentralized market and future business opportunities, the technology is still in its infancy. There are several challenges that need to be carefully addressed, and many significant opportunities that need to be seized.

NFTs, backed by blockchain technology and smart contracts, offer tremendous opportunities for the creative industry, though their presence has disrupted the market. However, NFTs also have limitations stemming from the underlying blockchain technology. Blockchain itself ensures trust within its distributed system by relying on computers (often called "miners") to solve complex mathematical problems. One major challenge in the evolution of blockchain technology is interoperability. Although blockchains provide robust and reliable solutions, the various types and variants of blockchains currently in existence often struggle to interact and communicate effectively. This implies that smart contracts or NFTs developed on one blockchain may not be compatible or recognized by another type of blockchain. This limitation restricts the blockchain's ability to grow and integrate with larger industry systems, such as finance, healthcare, or international business.

Blockchain and Smart Contracts are crucial foundations in realizing the vision of Web3.0. Blockchain provides transparency, security, and trust by storing data on a decentralized network, while Smart Contracts enable the automation of transactions and agreements in the digital world without intermediaries. The combination of these technologies could lay the foundation for a new era in the digital world, where interactions are more secure, transparent, and seamless.

The discussion in this paper begins with a presentation on other research (Section \ref{sec:penelitianterkait}). It then continues with an explanation of the architecture of the system developed (Section \ref{sec:arsitektur}). Based on this, we show the results of testing and analysis (Section \ref{sec:hasil}). Finally, conclusions from the research conducted are presented (Section \ref{sec:kesimpulan}).







