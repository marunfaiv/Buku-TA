% Ubah judul dan label berikut sesuai dengan yang diinginkan.
\section{Related Research}
\label{sec:penelitianterkait}

% Ubah paragraf-paragraf pada bagian ini sesuai dengan yang diinginkan.


Several other studies have been conducted, as formulated by \citet{Wang2021}, which suggest that NFTs can transform the digital or virtual asset market. The results of this research provide an in-depth understanding of NFT technology, its potential, and the challenges it faces. Additionally, research results from \citet{Zheng2020} have yielded knowledge in the form of a review on the latest smart contract technology, challenges in various aspects of creation, deployment, execution, and resolution of smart contracts, comparisons of several major smart contract platforms, and reviews on smart contract and blockchain technology. Cited from research previously written by \citet{Malik2023}, the study examines blockchain technology's impact on the creative industries, such as music, graphic design, gaming, and software. In this journal, they highlight how NFTs (non-fungible tokens) and smart contracts offer exciting opportunities for the creative industry. Although this technology has created significant excitement in the market, amidst the hype, real value emerges for the industry. Traditionally, creators in the creative industry often had to rely on powerful intermediaries to distribute and monetize their creations. However, with the advent of NFTs and smart contracts, creators can now be closer to their consumers/buyers of content. Additionally, this journal also explores the market share and "transaction costs" creators face when distributing their creative content and how smart contracts and NFTs can change market dynamics by reducing these costs.